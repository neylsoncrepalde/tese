\documentclass[a4paper, 12pt, openright, oneside, german, french, english, brazil]{abntex2}
\usepackage[brazil]{babel}
\usepackage{graphicx}
\usepackage[utf8]{inputenc}
\usepackage{wrapfig}
\usepackage{lscape}
\usepackage{rotating}
\usepackage{epstopdf}
\usepackage[alf]{abntex2cite}
\usepackage[a4paper, left=3cm, right=2cm, top=3cm, bottom=2cm]{geometry}
\usepackage{indentfirst}
\usepackage{longtable}
\usepackage{amsmath}
\usepackage{verbatim}
\usepackage{algorithm}
\floatname{algorithm}{Código}
\renewcommand{\listalgorithmname}{Lista de Códigos}
\usepackage{algpseudocode} %para escrever pseudo-algoritmos
%\algrenewcommand\algorithmicwhile{\textbf{Enquanto}}
%\algrenewcommand\algorithmicfor{\textbf{Para}}
%\algrenewcommand\algorithmicif{\textbf{Se}}
%\algrenewcommand\algorithmicthen{\textbf{então}}
%\algrenewcommand\algorithmicelse{\textbf{Do contrário}}
%\algrenewcommand\algorithmicdo{\textbf{faça}}
%\algrenewcommand\algorithmicfunction{\textbf{Função}}
%\algrenewcommand\algorithmicend{\textbf{termina}}
\usepackage{listings} %para escrever códigos
\pagestyle{plain}

\begin{document}

	%Anotações MULTILEVEL NETWORKS

	\section{Mercados como Redes Multinível}

	\citeonline{brailly2016market} estudam um mercado internacional entre organizações. De acordo com os autores, por trás das relações interorganizacionais há sempre laços entre indivíduos. Algumas organizações precisam de encontros interpessoais para iniciarem ações conjuntas ou parcerias. À medida que essas parcerias se repetem a relação se torna cada vez mais interorganizacional e cada vez menos interpessoal caminhando em direção a prescindir de encontros entre membros específicos. Os autores argumentam que, para um melhor entendimento dos fenômenos mercantis, deveria-se estudar as complexas articulações entre esses dois níveis de ação.

	Nos estudos em redes, ambos os níveis tem sido levados em conta embora um de cada vez. Ou os autores concentram-se no nível das organizações (e coloca sua atenção em laços como alianças comerciais, trocas, parcerias que afetam o desempenho e as chances de sobrevivência das empresas) ou no nível dos indivíduos (identificando redes relacionais informais como amizade, aconselhamento, colaboração, troca de recursos e informação, etc.). \citeonline{brailly2016market} argumentam que as atividades econômicas e os mercados são moldados pelos dois níveis que operam de maneira interdependente. ``Um negócio entre duas companhias, que é um laço interorganizacional, depende de relações interpessoais e vice versa. Relações econômicas como negócios entre duas organizações e relações informais entre seus membros são interdependentes'' \cite[p. 246]{brailly2016market}. Ambos os níveis são, portanto, superpostos e parcialmente aninhados.

	Considerar trasações mercantis como fenômenos multinível implica em duas hipóteses: (1) a \textit{hipótese de dependência estrutural horizontal} dentro dos dois níveis e (2) a \textit{hipótese da dependência estrutural vertical} entre os níveis. A primeira postula que atores em ambos os níveis agem em contexto social. A segunda postula que a rede relacional de um indivíduo depende da rede de sua companhia e vice versa. Ambos os níveis de ação estão parcialmente aninhados.

	Duas estratégias de análise são comummente mobilizadas . Para analisar a dependência horizontal em ambos os níveis, os autores propõe o uso de ERGM's pois o modelo contextualiza os laços internodais em sua vizinhança imeditada (e.g., centralidade, díades, tríades e outras estruturas mais complexas). Para analisar a dependência vertical, os autores partem de uma intuição de comum na ARS que consiste da transformação de redes \textit{2-mode} em redes \textit{1-mode}. Essa transformação aloca um laço entre organizações que possuam um membro em comum e aloca um laço entre indivíduos que participam de uma mesma organização. Os autores propõe, entretanto, uma nova abordagem para dar conta de ambas as dependências ao mesmo tempo.

	Para os autores ``nos mercados, indivíduos podem tirar vantagem da reputação de suas organizações. Do mesmo modo, uma organização pode tirar vantagem da popularidade de seus empregados (...)'' \cite[p. 250]{brailly2016market}. % Comentar sobre esse intuição

	% Explicar a tal da abordagem que começa na p 250.














\end{document}
