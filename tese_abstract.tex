\documentclass[a4paper, 12pt, openright, oneside, german, french, brazil, english, article]{abntex2}
\usepackage[brazil]{babel}
\usepackage{graphicx}
\usepackage[utf8]{inputenc}
\usepackage{graphicx}
\usepackage{wrapfig}
\usepackage{lscape}
\usepackage{rotating}
\usepackage{epstopdf}
\usepackage[alf]{abntex2cite}
\usepackage[a4paper, left=2cm, right=2cm, top=2cm, bottom=2cm]{geometry}
\usepackage{indentfirst}
\usepackage{longtable}
\pagestyle{plain}

%A Construção Social da Qualidade no Mercado da Música de Concerto
\titulo{The social construction of quality in a Brazilian classical music market -- PhD Thesis}
\autor{Neylson J. B. F. Crepalde (GIARS - UFMG) \and Dr. Silvio Salej Higgins (GIARS - UFMG) - Advisor}
\data{October, 2017}
%\instituicao{UNIVERSIDADE FEDERAL DE MINAS GERAIS
%	\par
%	Faculdade de Filososia e Ciências Humanas}
%\local{Belo Horizonte}
%\orientador{Dr. Silvio Salej Higgins (PPGS -- UFMG)}

%\coorientador{Dr. Emmanuel Lazega (Sciences Po, CSO -- Paris)}
%\preambulo{Projeto de Pesquisa apresentado ao Programa de Pós-Graduação em Sociologia da UFMG como requisito para ingresso no Programa Doutorado Sanduíche no Exterior (PDSE).}
%\tipotrabalho{Tese (doutorado)}


\begin{document}
	\textual
	\maketitle
	
	\section*{Abstract}
		The main goal of this investigation is to uncover a Brazilian classical music market and explain it. To do that, we will use a theoretical framework composed by a fusion of two economic sociology theories, one regarding quality as the main discipline of production markets \cite{white2002markets}, other stating the centrality of the State \cite{fligstein2002architecture}, the isomorphisms concept \cite{dimaggio1983iron} and the brand new theory of multilevel networks \cite{lazega2016multilevel}. Important applied research is also taken into account \cite{favereau2002markets,franccois2005monde,biencourt2002market}. 
		% Hipóteses - Esboço
		%(1) Qualidade é o centro de gravidade do mercado e emerge de sua estrutura em rede.
		%(2) Quanto maior a centralidade de um músico na rede interindividual, maior a sua influência sobre a formação do padrão de qualidade
		%(3) Quanto mais uma orquestra proporciona incentivos estruturais aos músicos, mais alto é o seu posicionamento no ranking de qualidade
		%(4) Quanto mais complexa for a estrutura organizacional da orquestra ou de sua mantenedora, mais alto é o seu posicionamento no ranking de qualidade
		%(5) Quanto maior o relacionamento de uma orquestra com o Estado, mais alto é o seu posicionamento no ranking de qualidade.
		Our main hypotheses are: (1) quality is the gravity center of the market and it emerges from the market's network structure; (2) the greater a musician's importance in the interindividual network (in terms of popularity, activity and freedom of action), the bigger his influence shaping the quality standard; (3) the more an orchestra provides structural incentives for the musicians, the better it will be positioned in quality ranking; (4) the more complex the organizational structure of an orchestra or its maintainer, the better it will be positioned in quality ranking; (5) the closest the relationship of an orchestra with the State, the better it will be positioned in quality ranking. This study will be conducted on a multilevel network perspective. We will collect network and attribute data from 5 orchestras in Belo Horizonte, Brazil and from their musicians using an online survey. We will also collect data from critics and orchestras' suppliers. To analyze the collected data and test our hypotheses, we will use ERGM's, blockmodeling and transformations of 2-mode to 1-mode networks. The combination of these techniques will allow us to analyze our data in a multilevel perspective. We will be able to measure to what extent orchestras can benefit from sharing musicians and how musicians can take advantage of the orchestras they play in order to build their perceived prestige.
		% Explicar aonde está o multinível
		
		
		
		
				
	\postextual
	
	
	\anexos
	
	\begin{table}[ht]
		\ibgetab{
			\centering
			\caption{Proposed Indicators}
			\label{indicadores}
		}
		{\begin{tabular}{|c|c|}
				
				\hline
				\textbf{Concept} & \textbf{Indicator} \\
				\hline
				&  Musicians perception \\
				Quality  & Public perception \\
				& Critics perception \\
				&  Average ticket price \\
				&  Number of concerts in a season \\
				&  Total financing \\
				&  Quality index (will be created by the author) \\
				\hline
				
				& Networks \\
				Individual interaction & Centrality measures  \\
				& Constraint      \\
				\hline
				& Country of origin  \\
				Individual Context & City of origin  \\
				& Formation institution \\
				& Professor    \\
				\hline
				& Average income  \\
				Incentive Structure & Income by specific function \\
				& Presentation as a soloist or in selected chamber groups \\
				\hline
				Management Complexity  & Number of boards/sections  \\
				& Distance (in levels) from the musician to the CEO \\
				\hline
				Interaction with the State  & Total financing from the State  \\
				& Number of partnership contracts \\
				\hline
				
				
			\end{tabular}
		}
		{\fonte{Elaborated by the author.}}
		
	\end{table}
	
	\citeoption{abnt-full-initials=yes}
	\bibliography{BIBDOUTORADO, abnt-options}
\end{document}