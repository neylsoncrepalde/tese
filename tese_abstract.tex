\documentclass[a4paper, 12pt, openright, oneside, german, french, brazil, english, article]{abntex2}
\usepackage[brazil]{babel}
\usepackage{graphicx}
\usepackage[utf8]{inputenc}
\usepackage{graphicx}
\usepackage{wrapfig}
\usepackage{lscape}
\usepackage{rotating}
\usepackage{epstopdf}
\usepackage[alf]{abntex2cite}
\usepackage[a4paper, left=2cm, right=2cm, top=2cm, bottom=2cm]{geometry}
\usepackage{indentfirst}
\usepackage{longtable}
\pagestyle{plain}

%A Construção Social da Qualidade no Mercado da Música de Concerto
\titulo{The social construction of quality in Brazilian Classical Music market}
\autor{Neylson J. B. F. Crepalde (GIARS - UFMG) \and Dr. Silvio Salej Higgins (GIARS - UFMG) \and Dr. Emmanuel Lazega (CSO - SciencesPo)}
\data{October, 2017}
%\instituicao{UNIVERSIDADE FEDERAL DE MINAS GERAIS
%	\par
%	Faculdade de Filososia e Ciências Humanas}
%\local{Belo Horizonte}
%\orientador{Dr. Silvio Salej Higgins (PPGS -- UFMG)}

%\coorientador{Dr. Emmanuel Lazega (Sciences Po, CSO -- Paris)}
%\preambulo{Projeto de Pesquisa apresentado ao Programa de Pós-Graduação em Sociologia da UFMG como requisito para ingresso no Programa Doutorado Sanduíche no Exterior (PDSE).}
%\tipotrabalho{Tese (doutorado)}


\begin{document}
	\textual
	\maketitle
	
	\section*{Abstract}
		The main goal of this investigation is to uncover the classical music market and explain it. To do that, we will use a theoretical framework composed by a fusion of two economic sociology theories \cite{white2002markets,fligstein2002architecture}, one concept \cite{dimaggio1983iron} and one network science brand new theory \cite{lazega2016multilevel}. Our main hypotheses are that (1) quality is the gravity center of the market and it emerges from the market's network, (2) musicians shape the current quality standard, (3) orchestras' incentive structure also shapes the quality standard, (4) the formal structure of the maintainer organizations is directly related to the orchestra's positioning in quality hierarchy and (5) the State shapes the quality ranking. This study will be conducted on a multilevel network perspective. We will collect data from 7 orchestras in Belo Horizonte, Brazil and from their musicians using an online survey. We will also look for critics and orchestras' suppliers. To analyze the collected data and test our hypotheses, we will use ERGM (P* models), blockmodeling and transformations on 2-mode to 1-mode networks. All these techniques combined allow us to analyze our data in a multilevel perspective.	
		
		
		
				
	\postextual
	
	
	\anexos
	
	\begin{table}[ht]
		\ibgetab{
			\centering
			\caption{Proposed Indicators}
			\label{indicadores}
		}
		{\begin{tabular}{|c|c|}
				
				\hline
				\textbf{Concept} & \textbf{Indicator} \\
				\hline
				&  Musicians perception \\
				Quality  &  Average ticket price \\
				&  Number of concerts in a season \\
				&  Total financing \\
				&  Quality index (new) \\
				\hline
				
				& Networks \\
				Individual interaction & Centrality measures  \\
				& Constraint      \\
				\hline
				& Country of origin  \\
				Individual Context & City of origin  \\
				& Formation institution \\
				& Professor    \\
				\hline
				& Average income  \\
				Incentive Structure & Income by specific function \\
				& Presentation as a soloist or in selected chamber groups \\
				\hline
				Management Complexity  & Number of boards/sections  \\
				& Level-distance from the musician to the CEO \\
				\hline
				Interaction with the State  & Total financing from the State  \\
				& Number of partnership contracts \\
				\hline
				
				
			\end{tabular}
		}
		{\fonte{Elaborated by the author.}}
		
	\end{table}
	
	\citeoption{abnt-full-initials=yes}
	\bibliography{BIBDOUTORADO, abnt-options}
\end{document}