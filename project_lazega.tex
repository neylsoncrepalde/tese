\documentclass[a4paper, 12pt, openright, oneside, german, french, brazil, english, article]{abntex2}
\usepackage[brazil]{babel}
\usepackage{graphicx}
\usepackage[utf8]{inputenc}
\usepackage{graphicx}
\usepackage{wrapfig}
\usepackage{lscape}
\usepackage{rotating}
\usepackage{epstopdf}
\usepackage[alf]{abntex2cite}
\usepackage[a4paper, left=2cm, right=2cm, top=2cm, bottom=2cm]{geometry}
\usepackage{indentfirst}
\usepackage{longtable}
\pagestyle{plain}

%A Construção Social da Qualidade no Mercado da Música de Concerto
\titulo{The social construction of quality in Brazilian Classical Music market}
\autor{Neylson J. B. F. Crepalde (GIARS - UFMG) \and Dr. Silvio Salej Higgins (GIARS - UFMG) \and Dr. Emmanuel Lazega (CSO - SciencesPo)}
\data{October, 2017}
%\instituicao{UNIVERSIDADE FEDERAL DE MINAS GERAIS
%	\par
%	Faculdade de Filososia e Ciências Humanas}
%\local{Belo Horizonte}
%\orientador{Dr. Silvio Salej Higgins (PPGS -- UFMG)}

%\coorientador{Dr. Emmanuel Lazega (Sciences Po, CSO -- Paris)}
%\preambulo{Projeto de Pesquisa apresentado ao Programa de Pós-Graduação em Sociologia da UFMG como requisito para ingresso no Programa Doutorado Sanduíche no Exterior (PDSE).}
%\tipotrabalho{Tese (doutorado)}


\begin{document}
	\textual
	\maketitle
	
	
	This research project aims to look into the market of classical music in the Southeast region of Brazil. The research is guided by two main questions: 1) Understanding that the orchestras' market does not operate as an usual market, how does it work?;  who are the actors involved and which are the actions that each one develops within the system? 2) Which are the conditions and social factors of the production of orchestral music quality standards, i.e., how this quality standard to which everyone is commited to is produced and sustained?
	
%	Este trabalho visa debruçar-se sobre o mercado da música de concerto na região Sudeste do Brasil. A investigação é norteada por duas principais perguntas: 1) Entendendo que o mercado das orquestras não opera como um mercado comum, como se dá seu funcionamento; quem são os atores envolvidos e quais são as ações que cada um desenvolve no sistema? 2)Quais são as condições e fatores sociais de produção de padrões de qualidade da música de concerto, ou seja, como é produzido e mantido o \textit{standard} de qualidade com o qual todos estão comprometidos?

	To develop our study, it is necessary to transcend the limits with which the economic theory encountered when addressing cultural products. In all it's history, economic theory has anchored his main discoveries in two assumptions. The first leads to the idea of \textit{homo economicus}, i.e., the rational actor who makes decisions aiming to boost profits and reduce losses. To do that, he has access to complete information and he is capable of processing them entirely. This assumption gives the economy the capacity of elaborating elegant models to explain people's choices and preferences but cannot incorporate a central part of social life: culture. Economic sociology has develop itself mainly addressing the gaps left by this assumption seeking to give account for how social norms, values, status systems and prestige influence economic action. Put another way, a sociological perspective makes possible to see relational factors that help to answer classical questions of economy while it brings the studies closer to reality even if it does not have such elegant models or such consistent logic \cite{hirsch1987dirty}. Economic sociology is worried with contextual elements of economic trades, i.e., to analyze how interactions between actors bring out a market and how this interactions regulate and control the markets.

	
%	Para desenvolvermos nosso estudo faz-se necessário transcender os limites com os quais a teoria econômica se deparou ao abordar produtos culturais. Ora, em todo o seu decurso, a teoria econômica tem ancorado seus principais achados em dois pressupostos. O primeiro remonta à ideia do \textit{homo economicus}, ou seja, o ator racional que toma decisões visando potencializar ganhos e diminuir perdas. Para isso, ele tem acesso à completude das informações de que precisa e é capaz de processá-las inteiramente. Esse pressuposto dá à economia a capacidade de elaborar modelos elegantes para explicar escolhas e preferências mas não consegue incorporar uma parte central da vida social, a saber, a cultura. A área da sociologia econômica desenvolveu-se, sobretudo, se debruçando sobre as lacunas deixadas por esse pressuposto buscando dar conta de como normas sociais, valores, sistemas de status e prestígio influenciam a ação econômica. Dito de outra forma, uma perspectiva sociológica possibilita enxergar fatores relacionais que ajudam a responder a perguntas clássicas da economia além de tornar o estudo mais próximo da realidade mesmo não tendo modelos tão elegantes e nem lógica tão consistente \cite{hirsch1987dirty}. A sociologia econômica está preocupada com elementos contextuais das trocas econômicas, ou seja, analisar como as interações entre os atores fazem emergir um mercado e como essas interações regulam e controlam os mercados.
	
	The second assumption is the general characterization of market products. The products are understood by four objective criteria, namely, their physical properties (which in this case are directly related to the quality of the concerned product), the date and the place where they are available and what determines their delivery within a certain universe, i.e., without uncertainties. The quality of a product, in this view, can be decomposed into a series of objective elements, i.e., clearly measurable and ordered. Moreover, in classical economic theory every good is considered a `` private good '' and therefore ``unique and rival'' in its consumption. To give an example, ``a cup of coffee, a sandwich, a shirt, a pair of shoes, a chair, etc., are exclusive property because you can prevent me to get them (\ldots); on the other hand, each of these goods is for exclusive consumption because in the moment that I enjoy it, no one else can enjoy it'' \cite[p. 29]{tolila2007cultura}. However, the cultural products, generally, are not unique; You can, for example, admire a beautiful historic building on the street without having to pay for it. Nor they are rival in consumption; the pleasure of attending a concert is not diminished by the presence of other people in the audience.
	
	
%	O segundo pressuposto consiste da caracterização geral dos produtos mercantis. As mercadorias são entendidas por meio de quatro critérios objetivos, a saber, suas propriedades físicas (as quais, nesse caso, estão diretamente relacionadas com a qualidade do produto em questão), a data e o local em que estão disponíveis e aquilo que condiciona sua entrega num universo certo, i.e., sem incertezas. A qualidade de um bem, nessa perspectiva, pode ser decomposta em uma série de elementos objetivos, i.e., claramente mensuráveis e hierarquizáveis. Além disso, na teoria econômica clássica todo bem é considerado um ``bem privado'' e, portanto, ``exclusivo e rival'' no consumo. Para citar um exemplo, ``um café, um sanduíche, uma camisa, um par de sapatos, uma cadeira, etc., são bens exclusivos porque é possível impedir-me de obtê-los (\ldots); por outro lado, cada um desses bens é de consumo exclusivo porque no momento em que o aproveito, nenhuma outra pessoa pode usufruí-lo'' \cite[p. 29]{tolila2007cultura}. Ora, os produtos culturais, de um modo geral, são não exclusivos; pode-se, por exemplo, admirar um belo edifício histórico na rua sem ter que pagar por isso. Tampouco são rivais no consumo; o prazer de assistir um concerto não é diminuído pela presença de outras pessoas no público.

	The cultural industry is also defined by its supply logic turned to production instead of consumption, unlike ordinary goods markets. To \citeonline[p. 32]{tolila2007cultura}, ``this supply logic characterizes well, among others, the action of public policies in terms of investiment, help, and the sustenance of cultural activities, from patrimony to live performance, and in terms of incentives to cultural practices''. In fact, States and public collectivities have shown growing interest in cultural industry, which can be verified through public policies, specialized administrations, resources allocation driven specifically to this industry and the emergence of a whole network of institutions and professionals in cultural industry, most of them financed by public resources \cite{tolila2007cultura}.

	
%	O setor cultural define-se, ainda, pela sua lógica de oferta voltada à produção, ao contrário dos mercados de bens comuns voltados ao consumo. Para \citeonline[p. 32]{tolila2007cultura}, ``essa lógica da oferta caracteriza bem, entre outras, a ação das políticas públicas em termos de investimento, de ajuda e de sustentação das atividades culturais, do patrimônio ao espetáculo ao vivo, e em termos de incentivos às práticas culturais''. De fato, os Estados e coletividades públicas tem demonstrado interesse crescente no setor cultural, o que pode ser verificado através das políticas públicas, das administrações especializadas, da alocação de recursos dirigidos especificamente ao setor e do surgimento de toda uma rede de instituições e profissionais atuantes no setor cultural, grande parte deles financiados por dinheiro público \cite{tolila2007cultura}.

	The symbolic value of cultural goods constitutes, for us, a central element in the comprehension of our study object, although, counteracting again the classic economic theory, it is not objective in its nature but relational and individual, i.e., it only exists when it is recognized by the individual in the moment of its consumption. For the symbolic value of a determined good to be recognized, there must be adequate cognitive structures to the comprehension and fruition of the good, in other words, mental schemes acquired by previous artistic education \cite{bourdieu2003amor}.
	
%	O valor simbólico dos bens culturais constitui, para nós, um elemento central na compreensão de nosso objeto de estudo, muito embora, contrariando novamente a teoria econômica clássica, não seja objetivo em sua natureza mas relacional e individual, i.e., só existe à medida que é reconhecido pelo indivíduo no momento de seu consumo. Para que o valor simbólico de um determinado bem seja reconhecido, é necessário que haja estruturas cognitivas apropriadas para a compreensão e fruição do bem, ou seja, esquemas mentais adquiridos por meio da educação artística prévia \cite{bourdieu2003amor}.
	
	The musical performances have still a peculiarity related to the nature of their existence in which lies much of the methodological difficulties that surround them. \citeonline{tolila2007cultura} explains:
	
	\begin{citacao}
		What is music? The writen score? No. The musicians of the orchestra? No. The conductor? Neither. In fact, it is almost impossible to define music as a ``thing'' (a table, a chair, a house, etc.) for it only exists \textit{de facto} in the moment that it is heard, i.e., in a relation with the listener\footnote{To the author, ``in fact, in the social world, the real mode of existence of most phenomena is also that of the relation between human beings'' \cite[p. 110]{tolila2007cultura}. Curiosly, in this premisse is based all the neostructural paradigm in sociology also known as ``relational perspective'' or ``social networks theory''.}. \cite[p. 109]{tolila2007cultura}
	\end{citacao}
	
%	As performances musicais possuem ainda uma particularidade quanto à natureza de sua existência na qual reside grande parte das dificuldades metodológicas que as cercam. \citeonline{tolila2007cultura} explica:
	
%	\begin{citacao}
%		O que é a música? A partitura escrita? Não. Os músicos que formam a orquestra? Não. O regente? Também não. Na verdade, é quase impossível definir a música como uma ``coisa'' (uma mesa, uma cadeira, uma casa, etc.) pois ela só existe de fato no momento em que é ouvida, isto é, em uma relação com o ouvinte\footnote{Para o autor, ``na verdade, no mundo social, o modo real de existência da maioria dos fenômenos é o da relação entre seres humanos'' \cite[p. 110]{tolila2007cultura}. Curiosamente, nessa premissa se baseia todo o paradigma neoestrutural na sociologia conhecido também como ``perspectiva relacional'' ou ``teoria das redes sociais''.}. \cite[p. 109]{tolila2007cultura}
%	\end{citacao}

	Thus, the music (as well as dance and theater, for instance) assumes a special mode of existence that involves the participation of all elements or actors mentioned, i.e., the score, the musicians, the conductor, etc., in the construction of its materiality which only exists (and therefore is only possible to be consumed) in the very moment of its hearing.	
	
%	Desse modo, a música (bem como a dança e o teatro, por exemplo) assume um modo especial de existência que envolve a participação de todos os elementos ou atores supracitados, a saber, partitura, músicos, regente, etc., na construção de sua materialidade que só existe (e portanto só é possível de ser consumida) no momento da escuta.

	\cite{karpik2009elements} treats the same object under the perspective of ``économie de singularités''. To this author, singular goods and services ``are unknown by neoclassical economic theory. They do not exist''\footnote{(\dots) ils sont donc méconnus par la théorie économique néo-classique. Ils n'existent pas.} \cite[p. 163]{karpik2009elements}. The singularities are flagged by the presence of \textit{good} in their characterization as a diferential in a comparison between qualities, i.e., the \textit{good} wine, the \textit{good} music, the \textit{good} orchestra, etc. 
	
%	\citeonline{karpik2009elements} trata o mesmo objeto sob a ótica da ``economia de singularidades''. Para esse autor, bens e serviços singulares ``são desconhecidos pela teoria econômica neoclássica. Eles não existem''\footnote{(\dots) ils sont donc méconnus par la théorie économique néo-classique. Ils n'existent pas.} \cite[p. 163]{karpik2009elements}. As singularidades são sinalizadas pela presença do \textit{bom} em sua caracterização como diferencial numa comparação entre qualidades, i.e., o bom vinho, a boa música, a boa orquestra, etc. 
	
	\begin{citacao}
		The singularities are goods and services \textit{structured, uncertain and incommensurable}. Those three characteristics \textit{combined} characterize all the singularities as unique, multiple and their material support prescinds of industrial production, once their symbolic power is maintained and, consequently, their capacity of receiving an indetermined number of particular interpretations\footnote{Les singularités sont des biens et services \textit{structurés, incertains et incommensurables.} Ces trois traits \textit{combinés} caractérisent toutes les singularités que'elles soient uniques, multiples ou que leurs supports matériels relèvent de la production industrielle, dès lors qu'est maintenu leur pouvoir symbolique et, par voie de conséquence, leur capacité à accueillir un nombre indéterminé d'interprétations particulières.}. \cite[p. 164]{karpik2009elements}
	\end{citacao}
	
%	\begin{citacao}
%		As singularidades são bens e serviços \textit{estruturados, incertos e incomensuráveis}. Essas três características \textit{combinadas} caracterizam todas as singularidades como sendo únicas, múltiplas e seu suporte material prescinde da produção industrial, uma vez que seja mantido o seu poder simbólico e, por conseguinte, sua capacidade de acolher um número indeterminado de interpretações particulares\footnote{Les singularités sont des biens et services \textit{structurés, incertains et incommensurables.} Ces trois traits \textit{combinés} caractérisent toutes les singularités que'elles soient uniques, multiples ou que leurs supports matériels relèvent de la production industrielle, dès lors qu'est maintenu leur pouvoir symbolique et, par voie de conséquence, leur capacité à accueillir un nombre indéterminé d'interprétations particulières.}. \cite[p. 164]{karpik2009elements}
%	\end{citacao}
	
	According to the network theory, the definition of ``good'' emerge from the relations within a netdom \cite{white2008}. We, therefore aim to investigate and explain the emergence of the quality standard with which the actors are commited to. This investigation puts a double problem. In the one hand we deal with interorganizational networks where the orchestras build relations (ties) with suppliers and other actors in a production system, and with intraorganizational networks where we can understand the social processes of status emergence and collective learning among musicians; on the other, we deal with multilevel structures where both orchestras and musicians build relations and trade.
	
	First, we will map the production network of orchestral music in three state capitals of Southeast region of Brazil, namely, Belo Horizonte (Minas Gerais State), São Paulo (São Paulo State) and Rio de Janeiro (Rio de Janeiro state). We will track back the firms that negotiate and cooperate with these symphony orchestras and also the cooperation between them in terms of exchanging resources and sharing guest artists (conductors and soloists).
	
	In Belo Horizonte, today, there are two major orchestras, namely, Minas Gerais Symphony Orchestra and Minas Gerais Philharmonic Orchestra. Both are professional orchestras but musicians of the former are public workers and musicians of the later have contracts that can be revoked any time. Their salary, therefore, is much higher. It is common sense among the audience in Minas Gerais capital, Belo Horizonte, that the Philharmonic Orchestra has a better performing quality than the other one. We could ask in a very objective way, why is that?
	
	The main orchestra in Brazil is São Paulo State Symphony Orchestra (OSESP) conducted by Mrs. Marin Alsop. This group was the first one in Brazil to go through a change in its administrative model going from an orchestra of public workers to hired musicians. Their musicians have the biggest salary and the orchestra has the highest budgett in Brazil. They are also known as the ``best'' brazilian orchestra in terms of artistic quality and the one with the best international exposure. Besides OSESP there are other two professional symphony orchestras in São Paulo, namely, São Paulo University Orchestra (OSUSP) and São Paulo City Orchestra.
	
	Rio de Janeiro has three professional symphony orchestras: Brazilian Symphony Orchestra (OSB), Petrobrás Symphony Orchestra (OPES) and City Theater Symphony Orchestra.
	
	After mapping the orchestral music production network, we also aim to investigate how social roles (identities) emerge for each actor. To do that, we will perform a stochastic blockmodel in order to build blocks (clusters) of actors who are structurally equivalent, i.e., actors who have the same relational profile and, therefore, perform the same ``roles'' within the network. We also aim toexplain the tie formation process. We will model the emergence of the network as a function of its endogenous configurations and of exogenous attributes such as annual budget, amount of guest artists, amount of non-brazilian musicians, type of contract, etc. To do that, we will use exponential random graph models or p* models \cite{lazega2014redes}.
	
	Finally, we can point three reasons why this research should be conducted at CSO:
	
	\begin{enumerate}
		\item[a.] professor Lazega has developed methods to analyze network data in multilevel structures \cite{lazega2016multilevel};
		\item[b.] professor Lazega occupies the front line in the developments and advances regarding research in interorganization and intraorganizational networks today;
		\item[c.] such collaboration will fortify the research network between France and Brazil.
	\end{enumerate}
	
	
	
	
	
	
	
	
	
	
\postextual
\citeoption{abnt-full-initials=yes}
\bibliography{BIBDOUTORADO, abnt-options}
\end{document}